\newenvironment{proofenum}{%                                     
	\begin{enumerate}
		\renewcommand\labelenumi{(\arabic{section}.\arabic{enumi})}}
	{\end{enumerate}}

\chapter{Summary and concluding remarks}
	\label{CH_summary}

In this work, we have presented a dynamic modal logic for reasoning about the relationship between an agent's beliefs, knowledge, and action, with a distinction between mere action and safe action expressible. We named this logic Dynamic Agent Safety Logic, or \DASL. Including this many modalities in one logic makes it very expressible and suitable for modelling a variety of realistic situations involving human-like agents. Additionally, the logic was developed with careful concern for realism. This departs from the typical thesis of modern lmodal ogic, which develops a logic with advanced formal properties without much regard for the realism of its foundational axiom schemas. By carefully considering the philosophical implications of our static base, the thesis made advances in reconnecting the worlds of epistemology and epistemic logic, and philosophy and modal logic more generally.

As a thesis of computer science and logic, we took care to use computational tools to validate our logical theses. We used the Coq Proof Assistant to mechanically check our proofs of soundness and completeness to a large degree, and similarly mechanized our case studies which validate \DASL's application to aviation safety. 

The proof of completeness mechanized a powerful aspect of modal logic from Sahlqvist and van Benthem. This dissertation, as far as we know, marks the first use of Coq for proving a modal logic's completeness via Sahlqvist theorem. To do this, we developed inductive predicates over the structure of axiom schemas corresponding to the definitions in Blackburn \etal\cite{modal}. This two-leveled approach to mechanization allowed us to prove theorems about the logic, and then drop down into the object language and prove theorems in the logic. This approach is extensible to other domains, where an object language with different atoms could be defined in Coq, while the upper level schema mechanization remains fixed, along with its results.

We mechanically checked our formalization of three cases of aviation mishaps, zeroing in on particular moments from the events where \DASL\ enables an inference from action to ignorance about safety-critical information. This inference was foreshadowed by the foundational models of agency in classical game theory, which infer good actions from assumptions of rationality and knowledge. A boolean manipulation of that classical inference yields the inference from bad action and assumption of rationality to an absence of knowledge. \DASL\ provides the formal logic for capturing this inference in a rich and realistic fashion.

The work, however, is not finished. This dissertation made advances along these lines, but did not fully explore the new terrain. We took care to establish a static base logic with epistemic and doxastic operators that was realistic, and in particular distinguished our static base from those of \SFive\ and $\mathcal{S}4$, as seen in much of the rest of the epistemic logic literature. However, these logics, and our own static base logic, potentially suffer from a serious problem not explored in the present work.

The potential problem is due to L\"ob in \cite{Lob}. We describe the problem here for future work. We note that the problem is under active investigation by researchers at the Machine Intelligence Research Institute\footnote{See their website at https://www.intelligence.org}, who kindly invited the author to a workshop introducing the problem. Some of there preliminary results on the problem are available on the footnoted website. They have named the problem L\"ob's Obstacle, or the L\"obian Obstacle. We describe it, and the problem it presents to modal logic of agency, including our own, below.

\section{L\"ob's Theorem}
\label{sec:lob_section}
L\"ob's Obstacle takes its name from Martin Hugo L\"ob, who answered one of L. Henkin's follow up questions to G\"odel's striking incompleteness theorems. The question was, \emph{what about formulas that assert their own provability, as opposed to unprovability?}\footnote{Not a direct quote.} L\"ob's response was as follows. A sufficiently powerful system, \emph{e.g.} Peano arithmetic, can prove that formulas asserting that their own provability implies that they are true only when Peano arithmetic actually proves the formula. It seems a little trivial. For formal systems capturing the reasoning of agents, it can lead to surprising results.

L\"ob's Theorem in provability logic is,
\begin{center}~\label{lob}
	\begin{equation}
	\Box(\Box\varphi \iimplies \varphi)\iimplies \Box\varphi.
	\end{equation}
\end{center}
The $\Box$ is interpreted as ``provability" in some formal system, particularly one at least as powerful as Peano arithmetic. If a modal operator involves the reasoning abilities of a human-like agent, then \emph{a fortiori} it is a formal system at least as powerful as Peano arithmetic. This presents the following problem. If that agent, in its own reasoning system, can deduce the soundness of its own system, then it's reasoning system is unsound. This is because $\varphi$ can be any formula, including $\bot$. The particular modal logic at risk, in our mind, is epistemic logic, or any doxastic logic with accurate reasoners.

L\"ob identified the following conditions of a formal system that would allow derivation of his theorem (which we present in modal form).
\begin{enumerate}
	\item $\Box(\varphi \iimplies \psi) \iimplies (\Box\varphi \iimplies \Box \psi)$\mbox{}\hfill Axiom K
	\item $\Box\varphi \iimplies \Box\Box\varphi$ \mbox{}\hfill Axiom 4
	\item From $\vdash \varphi$, infer $\Box\varphi$ \mbox{}\hfill Rule of Necessitation
\end{enumerate}

Items (1) and (3) are costants for all normal modal logics. There is a suppressed condition that L\"ob did not mention because it was unnecessary for the domain of arithmetic, but we must mention it here. The system must either admit of self-referential sentences or involve modal fixed points. Because systems of human-like reasoning must include self-referential sentences, at least in the form of Peano arithmetic, in order to remain human-like, this condition is satisfied for our concerns. 

Here we give a template derivation of L\"ob's Theorem, which we shall refer to below when describing how L\"ob's Obstacle corrupts various epistemic logics.
\begin{proof}
	$\\$
\begin{proofenum}
	\item $\Box(\Box\varphi \iimplies \varphi)$\mbox{}\dotfill Assumption
	\item $\Box(\psi \iiff (\Box\psi\iimplies \varphi))$\mbox{}\dotfill L\"ob Sentence\footnote{Sometimes referred as a Curry sentence after logician Haskell Curry.}
	\item $\Box(\Box\psi \iiff \Box(\Box\psi \iimplies \varphi))$\mbox{}\dotfill Axiom K
	\item $\Box(\Box\psi \iimplies \Box(\Box\psi \iimplies \varphi)))$\mbox{}\dotfill (1.3) Simplification of $\iiff$
	\item $\Box(\Box\psi \iimplies (\Box\Box\psi \iimplies \Box\varphi))$\mbox{}\dotfill (1.4) Axiom K
	\item $\Box(\Box\psi \iimplies \Box\Box\psi)$\mbox{}\dotfill Axiom 4
	\item $\Box(\Box\psi \iimplies \Box\varphi)$\mbox{}\dotfill (1.5), (1.6)
	\item $\Box(\Box\psi \iimplies \varphi)$\mbox{}\dotfill (1.7), (1.1)
	\item $\Box\psi$\mbox{}\dotfill (1.3), (1.8)
	\item $\Box\Box\psi$\mbox{}\dotfill (1.9), Axiom 4
	\item $\Box\Box\psi \iimplies \Box\varphi$\mbox{}\dotfill (1.8), Axiom K
	\item $\Box\varphi$\mbox{}\dotfill (1.10), (1.11)
\end{proofenum}\mbox{}\hfill $\mathcal{QED}$
\end{proof}

Mathematical and Provability logicians refer to the key components of this proof as L\"ob Conditions\cite{Boolos}. Identifying them in the proof above helps us identify which epistemic logics collide with L\"ob's Obstacle. Conversely, understanding how the L\"ob Conditions interact helps us construct epistemic logics that avoid the Obstacle.

The Conditions are:
\begin{enumerate}
	\item The L\"ob Sentence. A self-referential sentence, also formalizable as a modal fixed point.
	\item Axiom K. The standard distribution axiom of normal modal logics.
	\item Axiom 4. The axiom corresponding to a transitive frame relation.
	\item The rule of necessitation. Likewise a standard feature of normal modal logics.
\end{enumerate} 

The L\"ob Sentence is sometimes not mentioned as a Condition, because L\"ob's Theorem is typically studied in the context of mathematical logic or provability logic, where such self-referential expressiveness is known to exist. We point out, however, that humans are capable of reasoning about self-referential sentences, as well, and any advanced artificial agent will be able to do so, as well. 

Finally, we note the importance of L\"ob's Theorem's antecedent: $\Box(\Box\varphi \iimplies \varphi)$. Epistemic logics typically include the antecedent as a theorem, in which case L\"ob's Theorem will allow us to derive $\Box\varphi$ for all $\varphi$. This is why consistent mathematical systems at least as expressive as Peano arithmetic cannot prove their own consistency. 

We identify some candidate epistemic logics and, on the assumption that they capture human-like reasoning, show how they crash into L\"ob's Obstacle.

\section{Epistemic Logics that Crash}
\label{sec:crashing_logics}
\subsection{\SFive\ Epistemic Logic}
The most prominent epistemic logic in the literature, by far, is \SFive\ epistemic logic. \SFive\ epistemic logic is routinely presented as the logic of knowledge, and often serves as a static base for dynamic extensions to epistemic logic involving action and communication. Its characteristic axioms are:
\begin{eqnarray}
\Kns{i}(\varphi \iimplies \psi)\iimplies (\Kns{i}\varphi \iimplies \Kns{i}\psi)\\
\Kns{i}\varphi \iimplies \varphi\\
\tlnot\Kns{i}\varphi \iimplies \Kns{i}\tlnot\Kns{i}\varphi
\end{eqnarray}
Clearly (2) is Axiom K, and (3), troublingly, is the antecedent of L\"ob's Theorem, known as Axiom T. (4) is called the Negative Introspection axiom, or sometimes in philosophy circles, the Wisdom Axiom. Logicians call it Axiom 5. It is read, ``If $i$ does not know that $\varphi$, then she knows that she doesn't know it". Other than being clearly invalid for humans, this axiom and (3) allows us to derive,
\begin{equation*}
\Kns{i}\varphi \iimplies \Kns{i}\Kns{i}\varphi
\end{equation*}

\begin{proof}
	$\\$
	\begin{proofenum}
		\item $\tlnot\Kns{i}\tlnot\Kns{i}\varphi \iimplies \Kns{i}\varphi$\mbox{}\hfill Contrapositive of Axiom 5
		\item $\Kns{i}\tlnot\Kns{i}\tlnot\Kns{i}\varphi \iimplies \Kns{i}\Kns{i}\varphi$\mbox{}\hfill Rule of Necessitation on (5.1), Axiom K
		\item $\varphi \iimplies \tlnot\Kns{i}\tlnot \varphi$\mbox{}\hfill Axiom T, Contrapositive
		\item $\tlnot\Kns{i}\tlnot\varphi \iimplies \Kns{i}\tlnot\Kns{i}\tlnot\varphi$\mbox{}\hfill Axiom 5
		\item $\varphi \iimplies \Kns{i}\tlnot\Kns{i}\tlnot \varphi$\mbox{}\hfill (4.3), (4.4)
		\item $\Kns{i}\varphi \iimplies \Kns{i}\tlnot\Kns{i}\tlnot\Kns{i}\varphi$\mbox{}\hfill $\Kns{i}\varphi$/$\varphi$, (4.5)
		\item $\Kns{i}\varphi\iimplies\Kns{i}\Kns{i}\varphi$\mbox{}\hfill (4.2), (4.6)
		
	\end{proofenum}\mbox{}\hfill$\mathcal{QED}$
\end{proof}

Thus, \SFive\ satisfies L\"ob's three conditions, if we assume the presence of self-referential sentences possible, which we should. Therefore, with $\Kns{i}$ instead of $\Box$, the proof of L\"ob's Theorem is possible in this brand of \SFive. However, to make matters worse, the antecedent of L\"ob's Theorem is itself an axiom of S5. Therefore, $\Kns{i}\varphi$ is a theorem, for all $\varphi$.

We take this as a \emph{reductio ad absurdum} that \SFive\ epistemic logic cannot be a logic for reasoning about the knowledge of agents with expressive power beyond Peano arithmetic. Therefore, it cannot be a logic of knowledge for humans, or human-like agents.

\subsection{Hintikka's S4 Epistemic Logic}
\label{sec:hint_s4}
In Hintikka's 1967 \emph{Knowledge and Belief: A logic of the two notions}, he presented an epistemic logic for determining the validity and consistency of claims people make about knowledge and belief. He rejected out of hand the negative introspection axiom for knowledge, but chose to include positive introspection, which is formalized as $\Kns{i}\varphi\iimplies\Kns{i}\Kns{i}\varphi$. Clearly then, if Hintikka's epistemic system is meant for human-like reasoners who can express sentences like, ``If I know this sentence is true, then 1 + 1 = 2," then it crashes into L\"ob's Obstacle, with the extra bite of having the antecedent of L\"ob's Theorem as a theorem itself, and therefore, $\Kns{i}\varphi$ is also a theorem.

\subsection{Kraus and Lehman System}
\label{sec:kl}

In \cite{KrausLehman}, Kraus and Lehman tackle the issue of how to combine knowledge and belief in a single system remarked on in the conclusion of Halpern and Moses in \cite{HalpernMoses}. The project at the time was very similar to that addressed by this dissertation's static foundation: the development of a logic sufficiently realistic and expressive for reasoning about intelligent agents. 

They axiomatize knowledge and belief as follows.

\begin{table}[H]
	\begin{center}
		\begin{tabular}{| l r |}
			\hline
			$\Kns{i}(\varphi \iimplies \psi) \iimplies (\Kns{i}\varphi \iimplies \Kns{i}\psi)$ & Distribution of $\Kns{i}$ \\
			$\Kns{i}\varphi \iimplies \varphi$ & Truth \\
			$\tlnot\Kns{i}\varphi \iimplies \Kns{i}\tlnot\Kns{i}\varphi$ & Negative Introspection \\
			$\Bels{i}(\varphi \iimplies \psi) \iimplies (\Bels{i}\varphi \iimplies \Bels{i}\psi)$ & Distribution of $\Bels{i}$\\
			$\Bels{i}\varphi \iimplies \BPoss{i}\varphi$ & Belief Consistency \\
			%		$\Bels{i}\varphi \iimplies \Bels{i}\Bels{i}\varphi$ & Positive Belief Introspection \\
			%		$\tlnot\Bels{i}\varphi \iimplies \Bels{i}\tlnot\Bels{i}\varphi$ & Negative Belief Introspection\\
			$\Kns{i}\varphi \iimplies \Bels{i}\varphi$ & Knowledge implies Belief \\
			$\Bels{i}\varphi \iimplies \Kns{i}\Bels{i}\varphi$ & Conscious Belief\\
			%			$\Bels{i}\varphi \iimplies \Kns{i}\Bels{i}\varphi$ & Beliefs are Known\\
			From $\vdash \varphi$ and $\vdash \varphi \iimplies \psi$, infer $\vdash\psi$ & Modus Ponens\\
			From $\vdash \varphi$, infer $\vdash \Kns{i}\varphi$ & Necessitation of $\Kns{i}$\\
			\hline
		\end{tabular}
		\caption{Logic of Kraus and Lehman}~\label{KL}
	\end{center}
\end{table}

The key differences with \DASL\ are that knowledge is \SFive\ and the axiom of Conscious Belief, which reverses the composition of knowledge and belief under belief relative to \DASL's Evidential Restraint axiom. From the above axiom schemas, it follows that belief is a regular $\mathcal{KD}45$ operator.

In their article, and in Meyer and van der Hoek's \cite{MeyerHoek}, they show that $\Bels{i}(\Bels{i}\varphi \iimplies \varphi)$ is a theorem. Therefore, this system suffers from a particularly bad collision with L\"ob: Both the knowledge and belief modalities lead to inconsistency for intelligent agenst capable of self-referential reasoning. 

\section{\DASL}
\label{sec:crap}

\begin{table}[H]
	\begin{center}
		\begin{tabular}{| l r |}
			\hline
			$\Kns{i}(\varphi \iimplies \psi) \iimplies (\Kns{i}\varphi \iimplies \Kns{i}\psi)$ & Distribution of $\Kns{i}$ \\
			$\Kns{i}\varphi \iimplies \varphi$ & Truth \\
			$\Bels{i}(\varphi \iimplies \psi) \iimplies (\Bels{i}\varphi \iimplies \Bels{i}\psi)$ & Distribution of $\Bels{i}$\\
			$\Bels{i}\varphi \iimplies \BPoss{i}\varphi$ & Belief Consistency \\
			%		$\Bels{i}\varphi \iimplies \Bels{i}\Bels{i}\varphi$ & Positive Belief Introspection \\
			%		$\tlnot\Bels{i}\varphi \iimplies \Bels{i}\tlnot\Bels{i}\varphi$ & Negative Belief Introspection\\
			$\Kns{i}\varphi \iimplies \Bels{i}\varphi$ & Knowledge implies Belief \\
			$\Bels{i}\varphi \iimplies \Bels{i}\Kns{i}\varphi$ & Evidential Restraint\\
			%			$\Bels{i}\varphi \iimplies \Kns{i}\Bels{i}\varphi$ & Beliefs are Known\\
			From $\vdash \varphi$ and $\vdash \varphi \iimplies \psi$, infer $\vdash\psi$ & Modus Ponens\\
			From $\vdash \varphi$, infer $\vdash \Kns{i}\varphi$ & Necessitation of $\Kns{i}$\\
			\hline
		\end{tabular}
		\caption{Logic of \DASL}~\label{GC_agent}
	\end{center}
\end{table}

%
%The operators in this logic are reduced in power relative to other logics of knowledge and belief. Here, knowledge is true, belief is consistent, knowledge implies belief, beliefs are subjectively justified, and both operators are normal modal operators. Knowledge lacks both epistemic introspection properties, while belief has positive introspection. As with all normal epistemic operators, these suffer from the problem of logical omniscience, which this paper does not attempt to solve. Rather, we shall show that in the instance of knowledge, the L\"obian Obstacle is avoided. The positive introspection property of belief is a theorem, however, so L\"ob's Theorem will be derivable for the belief operator.
\begin{theorem}[Positive Belief Introspection]~\label{belief_posint}
	$\Bels{i}\varphi \iimplies \Bels{i}\Bels{i}\varphi$
\end{theorem}
\begin{proof}
	$\\$
\begin{proofenum}
		\item $\Bels{i}\varphi \iimplies \Bels{i}\Kns{i}\varphi$\mbox{}\hfill ER Axiom
		\item $\Bels{i}\Kns{i}\varphi \iimplies \Bels{i}\Bels{i}\varphi$\mbox{}\hfill KiB Axiom + Necessitation of $\Bels{i}$\footnote{This rule can be derived from Necessitation of $\Kns{i}$ and KiB Axiom.}
		\item $\Bels{i}\varphi \iimplies \Bels{i}\Bels{i}\varphi$\mbox{}\hfill (3.1), (3.2)
	\end{proofenum}
\end{proof}\mbox{}\hfill $\mathcal{QED}$

The logic defined by these axiom schemas and inference rules avoids L\"ob's Obstacle for $\Kns{i}$, as it is no longer has the positive introspection property.

\begin{figure}[H]
	\begin{center}
		\begin{tikzpicture}[->,>=stealth',shorten >=1pt,auto,node distance=3cm,
		thick,base node/.style={circle,draw,minimum size=35pt}]
		
		\node[base node] (w) {\begin{tabular}{c}
			$w:  p$ \\ $\Kns{i}p$ \\ $\tlnot\Kns{i}\Kns{i}p$
			\end{tabular}};
		\node[base node] (v) [right of=w] {\begin{tabular}{c}
			$v:  p$ \\ $\tlnot\Kns{i}p $
			\end{tabular}};
		\node[base node] (u) [right of=v] {$u: \tlnot p$};
		\path[]
		(w) edge node[above] {$\Rel{k}^i$} (v)
		edge [loop left] node {$\Rel{k}^i$} (w)
		(v) 
		%		edge node[below] {} (w)
		%	\draw [->] (v) edge[in=5,out=355,loop] node[right] {$R$} (v)
		edge [<-, loop above] node {$\Rel{k}^i$} (v)
		edge node[above] {$\Rel{k}^i$} (u)
		(u)  edge [<-, loop right] node {$\Rel{k}^i$} (u);
		\end{tikzpicture}
	\end{center}
	\caption{A counterexample to $\Kns{i}\varphi\iimplies\Kns{i}\Kns{i}\varphi$.}
\end{figure}
%
%This logic also avoids the disaster of L\"ob's Theorem for belief, because $\Bels{i}(\Bels{i}\varphi \iimplies \varphi)$ is no longer a theorem.
%
%\begin{figure}[H]
%	\begin{center}
%		\begin{tikzpicture}[->,>=stealth',shorten >=1pt,auto,node distance=3.4cm,
%		thick,base node/.style={circle,draw,minimum size=35pt}]
%		
%		\node[base node] (w) {\begin{tabular}{c}
%			$w:  p$ \\ $\BPoss{i}(\Bels{i}p \tland \tlnot p)$
%			\end{tabular}};
%		\node[base node] (v) [right of=w] {\begin{tabular}{c}
%			$v:  \tlnot p$  \\
%			$\Bels{i}p$
%			\end{tabular}};
%		\node[base node] (u) [right of=v] {$u:  p$};
%		\path[]
%		(w) edge node[above] {$\Rel{k,b}^i$} (v)
%		edge [loop left] node {$\Rel{k}^i$} (w)
%		(v) 
%		%		edge node[below] {} (w)
%		%	\draw [->] (v) edge[in=5,out=355,loop] node[right] {$R$} (v)
%		edge [<-, loop above] node {$\Rel{k}^i$} (v)
%		edge node[above] {$\Rel{k,b}^i$} (u)
%		(u)  edge [<-, loop right] node {$\Rel{k,b}^i$} (u);
%		\end{tikzpicture}
%	\end{center}
%	\caption{A counterexample to $\Bels{i}(\Bels{i}\varphi \iimplies \varphi)$.}~\label{not_conceited}
%\end{figure}
%
%Therefore, L\"ob's Theorem has the same effect for belief as it does for Peano arithmetic: $i$ believes that her belief is accurate only for propositions that she actually believes. Her belief in her own accuracy is not a general claim. In this light, its effect is mitigated, in the same way it is for arithmetic. Because L\"ob's antecedent is not valid for belief, $i$ is not what Smullyan calls a \emph{conceited} reasoner. With L\"ob's Theorem for belief, $i$ constitutes what Smullyan calls a \emph{modest} reasoner.
%
%How shall we interpret the $\BPoss{i}$ operator? Figure \ref{not_conceited} could be read ``it is consistent with what $i$ believes that $i$ falsely believes that $p$". It is somewhat standard to interpret knowledge and belief operators in this way: ``...consistent with what $i$ knows/believes".  We do not endorse this interpretation due to the smell of circularity hanging around it. We prefer the interpretation ``$i$ considers it possible that..." for $\BPoss{i}$ and ``$i$ has evidence that..." for $\Poss{i}$. 
%
%If we consider the contrapositive of L\"ob's Theorem for belief, it states $\tlnot \Bels{i} \varphi \iimplies \tlnot\Bels{i}(\Bels{i} \varphi \iimplies \varphi)$. Indeed, this is G\"odel's Second Incompleteness Theorem applied to belief: ``if $i$ does not believe $\varphi$, then $i$ does not believe that believing $\varphi$ implies its truth." If we substitute in $BPoss{i}$ for $\tlnot\Bels{i}\tlnot$ and push the negation into the implication, the consequent is $\BPoss{i}(\Bels{i}\varphi \tland \tlnot \varphi)$, which is the formula illustrated in Figure \ref{not_conceited}. We read this as ``$i$ considers it possible that she believe $p$ but $p$ be false. This is perhaps a bit strange. However, the interpretations we propose have the following benefit.
%
%Recall that the contrapositive of the Knowledge implies Belief axiom is $\BPoss{i}\varphi \iimplies \Poss{i}\varphi$. Our interpretation reads this as, ``$i$ considers $\varphi$ to be possible only if she has evidence that $\varphi$."  This may not be valid for how humans \emph{in fact} form beliefs, but it provides a normative dimension to the logic and some modest level of idealization. The agent's modeled by this logic are human-like, and maybe some particular humans actually abide by these axioms. We call these agents Grounded-Coherent agents, because their beliefs are grounded in evidence, and coherent because they avoid L\"ob's Obstacle.
%
%The belief operator narrowly avoids L\"ob's Obstacle. 
Doxastic logic typically includes as an axiom of Belief Consistency, which corresponds to a serial doxastic possibility relation. Crucially, this axiom $\Bels{i}\varphi\iimplies\BPoss{i}\varphi$ is equivalent to $\tlnot(\Bels{i}\varphi \tland \Bels{i}\tlnot\varphi)$, which is furthmore equivalent to $\tlnot\Bels{i}(\varphi \tland \tlnot \varphi)$, which results in a disaster.

\begin{theorem}[Consistency Disaster]~\label{no_bel_cons}
	If $\tlnot\Bels{i}(\varphi \tland \tlnot \varphi)$ and\\ $\Bels{i}(\Bels{i}\varphi \iimplies\varphi)\iimplies \Bels{i}\varphi$ are theorems, then $\Bels{i}(\varphi \tland \tlnot \varphi)$.
\end{theorem}
\begin{proof}$\\$
\begin{proofenum}
		\item $\tlnot\Bels{i}(\varphi \tland \tlnot \varphi)$\mbox{}\hfill Belief is Consistent
		\item $\Bels{i}(\varphi\tland\tlnot\varphi)\iimplies (\varphi \tland \tlnot \varphi)$\mbox{}\hfill (3.1)
		\item $\Bels{i}(\Bels{i}(\varphi \tland \tlnot \varphi)\iimplies (\varphi \tland \tlnot \varphi))$\mbox{}\hfill Necessitation of $\Bels{i}$
		\item $\Bels{i}(\Bels{i}(\varphi \tland \tlnot \varphi) \iimplies (\varphi \tland \tlnot \varphi))\iimplies \Bels{i}(\varphi \tland \tlnot \varphi)$\mbox{}\hfill L\"ob's Theorem
		\item $\Bels{i}(\varphi \tland \tlnot \varphi)$\mbox{}\hfill (3.3), (3.4)
	\end{proofenum}
\end{proof}

Thus, with Belief Consistency, L\"ob's Theorem, and Theorem \ref{no_bel_cons}, it follows that:

\begin{theorem}[Beliefs Inconsistent]~\label{crap}
	For all $\varphi$, $\Bels{i}\varphi$.
\end{theorem}

\begin{proof}
	This follows from Theorem \ref{no_bel_cons} and $\tlnot\Bels{i}(\varphi \tland \tlnot \varphi)$.
\end{proof}\mbox{}\hfill $\mathcal{QED}$

Therefore, the logic is inconsistent. Our only recourse, if we allow self-reference, is to severely relax the axiom schemas.

%However, so long as it is merely not the case that $i$ believes $\varphi$ and that she believes $\tlnot \varphi$, we avoid disaster. It is worth exploring the cost of avoiding L\"ob's Obstacle. Does this epistemic logic still characterize something resembling a realistic logic of human-like knowledge? It still suffers from the problem of logical omniscience, for both knowledge and belief. This is unavoidable for a normal modal logic. It maintains the critical axiom that knowledge is true. 

One might wonder whether it would be acceptable to abandon the Truth Axiom for knowledge and allow L\"ob's Theorem to hold for the knowledge operator. This would introduce more modesty to the notion of knowledge, where a human-like agent knows that her knowledge is true only for those propositions that she actually knows, but not in the general sense. What would this mean for epistemology? A false proposition would no longer imply a lack of knowledge. It is an unfamiliar notion, perhaps worth exploring. Relaxing the Truth Axiom allows positive and negative introspection to live harmoniously with self-reference. We leave this for future work to explore.

\section{Avoiding L\"ob}
\label{sec:avoiding_lob}
We leave the reader with the following logic that avoids the L\"obian Obstacle while retaining the inference of safety-critical information that is missing.

\begin{table}[H]
	\begin{center}
		\begin{tabular}{| l r |}
			\hline
			$\Kns{i}(\varphi \iimplies \psi) \iimplies (\Kns{i}\varphi \iimplies \Kns{i}\psi)$ & Distribution of $\Kns{i}$ \\
			$\Kns{i}\varphi \iimplies \varphi$ & Truth \\
			$\Bels{i}(\varphi \iimplies \psi) \iimplies (\Bels{i}\varphi \iimplies \Bels{i}\psi)$ & Distribution of $\Bels{i}$\\
			$\Bels{i}\varphi \iimplies \BPoss{i}\varphi$ & Belief Consistency \\
			%		$\Bels{i}\varphi \iimplies \Bels{i}\Bels{i}\varphi$ & Positive Belief Introspection \\
			%		$\tlnot\Bels{i}\varphi \iimplies \Bels{i}\tlnot\Bels{i}\varphi$ & Negative Belief Introspection\\
			$\Kns{i}\varphi \iimplies \Bels{i}\varphi$ & Knowledge implies Belief \\
			$\Bels{i}\varphi \iimplies \Poss{i}\Kns{i}\varphi$ & Weak Evidential Restraint\\
			%			$\Bels{i}\varphi \iimplies \Kns{i}\Bels{i}\varphi$ & Beliefs are Known\\
			From $\vdash \varphi$ and $\vdash \varphi \iimplies \psi$, infer $\vdash\psi$ & Modus Ponens\\
			From $\vdash \varphi$, infer $\vdash \Kns{i}\varphi$ & Necessitation of $\Kns{i}$\\
			\hline
		\end{tabular}
		\caption{Logic of Grounded Coherent Epistemic Agents}~\label{GC_agent}
	\end{center}
\end{table}

We call it a logic of Grounded Coherent Epistemic Agents, because their beliefs are grounded by objectively available evidence, and their beliefs are coherent. Neither the belief operator nor the knowledge operator is susceptible to L\"ob's Obstacle, as L\"ob's Theorem is not derivable in the system.

In the counterexample below, we see that the belief operator lacks positive introspection.

\begin{figure}[H]
	\begin{center}
		\begin{tikzpicture}[->,>=stealth',shorten >=1pt,auto,node distance=3cm,
		thick,base node/.style={circle,draw,minimum size=35pt}]
		
		\node[base node] (w) {\begin{tabular}{c}
			$w:  p$ \\ $\Bels{i}p$ \\ $\tlnot\Bels{i}\Bels{i}p$
			\end{tabular}};
		\node[base node] (v) [right of=w] {\begin{tabular}{c}
			$v:  p$ \\ $\tlnot\Bels{i}p $
			\end{tabular}};
		\node[base node] (u) [right of=v] {$u: \tlnot p$};
		\path[]
		(w) edge node[above] {$\Rel{b}^i$} (v)
		(v) 
		%		edge node[below] {} (w)
		%	\draw [->] (v) edge[in=5,out=355,loop] node[right] {$R$} (v)
		edge node[above] {$\Rel{b}^i$} (u)
		(u) edge [<-, loop above] node {$\Rel{b}^i$} (u);
		
		\end{tikzpicture}
	\end{center}
	\caption{A counterexample to $\Bels{i}\varphi\iimplies\Bels{i}\Bels{i}\varphi$.}
\end{figure}

Without positive belief introspection, L\"ob's Theorem for belief is no longer derivable, and therefore this (very weak) epistemic logic avoids L\"ob's Obstacle. The resulting logic is still in the Sahlqvist class, so it is sound and complete.

\begin{tcolorbox}
	\begin{lstlisting}[language=Coq]
	Theorem weak_evidential_restraint_is_sahlqvist: 
		forall (phi : prop) (a : DASL.Agents),
			sahlqvist_formula 
			(SB a (SProp phi) =s=> \ (SK a (\ (SK a (SProp phi))))).
	Proof.
		intros; sahlqvist_reduce. 
	Qed.
	\end{lstlisting}	
	
\end{tcolorbox}

We provide the following counterexample to L\"ob's Theorem.

\begin{figure}[H]
	\begin{center}
		\begin{tikzpicture}[->,>=stealth',shorten >=1pt,auto,node distance=4.5 cm,
		thick,base node/.style={circle,draw,minimum size=35pt}]
		
		\node[base node] (w) {\begin{tabular}{c}
			$w:  \Bels{i}(\Bels{i}p \iimplies p)$ \\ $\tlnot\Bels{i}p$
			\end{tabular}};
		\node[base node] (v) [right of=w] {\begin{tabular}{c}
			$v:  \Bels{i}p \iimplies p $ \\ $\tlnot p $
			\end{tabular}};
%		\node[base node] (u) [right of=v] {$u: \tlnot p$};
		\path[]
		(w) edge node[above] {$\Rel{b}^i$} (v)
		(v) 
		%		edge node[below] {} (w)
		%	\draw [->] (v) edge[in=5,out=355,loop] node[right] {$R$} (v)
		edge [<-, loop above] node {$\Rel{b}^i$} (v);
%		(u) edge [<-, loop above] node {$\Rel{b}^i$} (u);
		
		\end{tikzpicture}
	\end{center}
	\caption{A counterexample to $\Bels{i}(\Bels{i}\varphi \iimplies \varphi) \iimplies \Bels{i}\varphi$.}
\end{figure}

We assume in world $w$ that $\Bels{i}(\Bels{i}p \iimplies p)$ holds. From this it follows that in all doxastically accessible worlds, \emph{e.g.} $v$, $\Bels{i}p \iimplies p$ holds. This holds for $v$ when it has reflexive access only to itself, and $\tlnot p$ is the case. Because $\tlnot p$ is the case $v$, $\tlnot \Bels{i}p$ is the case at $w$, concluding the counterexample.

The logic also includes a sort of weakened positive introspection theorem about knowledge, from the Knowledge implies Belief axiom and Weak Evidential Restraint. It is $\Kns{i}\varphi \iimplies \Poss{i}\Kns{i}\varphi$, which we read as ``$i$ knows that $\varphi$ only if she has objectively available evidence that she knows $\varphi$". This seems intuitive, and does not allow L\"ob's theorem to destroy the integrity of knowledge. It is perhaps a satisfying compromise for those who find positive introspection about knowledge to be intuitive. The contrapositive of this weak positive introspection formula, $\Kns{i}\tlnot\Kns{i}\varphi \iimplies \tlnot\Kns{i}\varphi$ is an instance of the T axiom, so it turns out to have been a theorem all along anyway, for any epistemic logic with the Truth axiom for knowledge.

This logic allows us to make the key inference discussed in this thesis, when it serves as a static foundation for weakened \DASL.

\begin{theorem}[Weakened \DASL\ Inference of Safety-Critical Information]$\\$

$(\PalPos{A, \laa}true \tland \tlnot pre_s(\laa) )\iimplies \tlnot \Kns{i}pre_s(\laa)\tland \tlnot\Kns{i}\tlnot \Kns{i}pre_s(\laa)$.
\end{theorem}
\begin{proof}
	Assuming $w\models\PalPos{A,\laa}true$, it follows that $\Bels{i}pre_s(\laa)$, from minimum rationality. From weak evidential restraint, it follows that $\Poss{i}\Kns{i}pre_s(\laa)$, which is equivalent to $\tlnot \Kns{i}\tlnot\Kns{i}pre_s(\laa)$. From $\tlnot pre_s(\laa)$, it follows from the Truth axiom that $\tlnot \Kns{i}pre_s(\laa)$.
\end{proof}

Further work must be done to explore this weakened \DASL. However, this dissertation is concluded.